\documentclass[10pt,journal,compsoc]{IEEEtran}

% *** CITATION PACKAGES ***
%
\ifCLASSOPTIONcompsoc
  % IEEE Computer Society needs nocompress option
  % requires cite.sty v4.0 or later (November 2003)
  \usepackage[nocompress]{cite}
\else
  % normal IEEE
  \usepackage{cite}
\fi

\usepackage{graphicx}
\usepackage{amssymb}
\usepackage{listings}
\usepackage{color}

\definecolor{dkgreen}{rgb}{0,0.6,0}
\definecolor{gray}{rgb}{0.5,0.5,0.5}
\definecolor{mauve}{rgb}{0.58,0,0.82}

\lstset{frame=tb,
  language=Python,
  aboveskip=3mm,
  belowskip=3mm,
  showstringspaces=false,
  columns=flexible,
  basicstyle={\small\ttfamily},
  numbers=none,
  numberstyle=\tiny\color{gray},
  keywordstyle=\color{blue},
  commentstyle=\color{dkgreen},
  stringstyle=\color{mauve},
  breaklines=true,
  breakatwhitespace=true,
  tabsize=3,
  morekeywords={if, in, or, then, enter, into, end, command, delete, from, not},
  xleftmargin=\dimexpr 1em+3pt\relax,
  linewidth=\dimexpr \linewidth-3pt\relax
}

\lstnewenvironment{itemlisting}[1][]
 {%
  \mbox{}
  \vspace*{-\baselineskip}
  \lstset{
    xleftmargin=\leftmargin,
    linewidth=\linewidth,
    #1
  }%
 }
 {}
 
% correct bad hyphenation here
\hyphenation{op-tical net-works semi-conduc-tor}

\renewcommand{\arraystretch}{1.5}

\begin{document}

\title{CS 6600 Project Report: Unmanned Aircraft System}

\author{Matthew Whitesides and Bruce M. McMillin \\
Department of Computer Science \\
Missouri University of Science and Technology, Rolla, MO 65409-0350} 

% The paper headers
\markboth{Missouri S\&T COMP\_SCI 6600: Formal Methods in CmpSec, Fall~2021}%
{Shell \MakeLowercase{\textit{et al.}}: Bare Demo of IEEEtran.cls for Computer Society Journals}
 
% The paper abstract
\IEEEtitleabstractindextext{%
\begin{abstract}
The abstract goes here.
\end{abstract}}

% make the title area
\maketitle

\IEEEpeerreviewmaketitle

%-------------------------------------------------------------------------

\section{Infrastructure}

\subsection{Infrastructure Description}

\IEEEPARstart{I}{n} its simplest form, an unmanned aircraft system (UAS), is an aircraft system that operates without an onboard pilot. 
UAS are either controlled remotely through some form of wireless radio communication, semi-autonomously in conjunction with a remote pilot, or fully autonomously using some form of computational intelligence as navigation. 
These intelligent crewless vehicles have many potential uses in the defense sector, including surveillance, strategic mission execution, aerial sustainability support, and training systems, to name a few. 
These aircraft fall under the umbrella of cyber-physical systems (CPS), merging the intelligent navigation processes, system health monitoring, and communication with the physical mobile ariel vehicle.

We will design a proposed infrastructure security policy for the Boeing MQ-25 unmanned aircraft system, but it will also apply to similar mission support drones. 
The MQ-25 is an unmanned aircraft system designed for the U.S. Navy, and it provides autonomous refueling capability for the Boeing F/A-18 Super Hornet, Boeing EA-18G Growler, and Lockheed Martin F-35C fighters.
This capability extends the combat range of the supported aircraft, seamlessly and semi-anonymously navigating to the plane, refueling, and returning to base. 
MQ-25 is the first unmanned aircraft to support aerial refueling another aircraft and is currently in the flight test phase of development [1], making it the perfect system to analyze security impacts for current and future unmanned aircraft systems. 

\subsubsection{Infrastructure Security Policy}

\begin{table}[]
  \caption{Description of rights over objects in the UAS.}
  \begin{tabular}{ll}
  \hline
  \textbf{Right}       & \textbf{Description}                               \\ 
  \hline
  \textit{Owns (O)}    & The owner of the given object.                     \\
  \textit{Read (R)}    & Can observe the given object.                      \\
  \textit{Write (W)}   & Can modify the given object.                       \\
  \textit{Execute (E)} & Can execute the functionality of the given object. \\
  \textit{Grant (G)}   & Can grant a given right to another subject.        \\
  \textit{Control (C)} & Can control a given system object.                 \\
  \textit{Delete (C)}  & Can delete a given .                 \\
  \hline
  \end{tabular}
\end{table}

\begin{table*}[]
  \caption{Description of actor subject roles during a UAS refueling mission.}
  \begin{tabular}{ll}
  \hline
  \textbf{Subject}                           & \textbf{Description}                                                                                       \\ 
  \hline
  \textit{Pilot Commander (PC)}              & The primary remote pilot of the UAS during the mission.                                                    \\
  \textit{Mission Commander (MC)}            & Responsible for final planning and decision making during the UAS mission.                                 \\
  \textit{Instructor Pilot (IP)}             & Assists the Pilot Commander and can pilot the UAS if given permission from the PC or MC.                   \\
  \textit{Quality Assurance Evaluators (QA)} & Responsible for pre-flight operational checks, in-flight analysis, and post-flight evaluation.             \\
  \textit{Maintenance Crew (MC)}             & Handles work orders created by the QA and FDA, responsible for the maintenance of the UAS.                 \\
  \textit{Flight Data Admin (FDA)}           & Handles and analyses all mission flight data.                                                              \\
  \hline
  \end{tabular}
\end{table*}

\begin{table*}[]
  \centering
  \caption{UAS Refueling Mission Access Control Matrix}
  \label{tab:my-table}
  \resizebox{\textwidth}{!}{
    \begin{tabular}{lllllllllllll}
    \hline
    \multicolumn{1}{l|}{\textbf{}}                                    & \textit{PC}   & \textit{MC}   & \textit{IP}   & \textit{QA}   & \textit{MC}   & \textit{FDA}  & \textbf{RNC}  & \textbf{ANC}  & \textbf{NP}   & \textbf{RO}   & \textbf{FED}  & \textbf{RTD}  \\
    \hline
    \multicolumn{1}{l|}{\textit{Pilot Commander (PC)}}                & O,R,W         & R,W           & R,W           & R,W           & R,W           & R,W           & O,R,W,E,G,C   & O,R,W,E,G,C   & O,R,W,E,G,C   & O,R,W,E,G,C   & R,W,E,G,C     & R,W,E,G,C     \\
    \multicolumn{1}{l|}{\textit{Mission Commander (MC)}}              & R             & O,R,W         & R,W           & R             & R             & R             & R,W,E,G,C     & R,W,E,G,C     & R,W,E,G,C     & R,W,E,G,C     & R,W,E,G,C     & R,W,E,G,C     \\
    \multicolumn{1}{l|}{\textit{Instructor Pilot (IP)}}               & R             & $\varnothing$ & O,R,W         & $\varnothing$ & $\varnothing$ & $\varnothing$ & R,W,E,C       & R,W,E,C       & R,W,E,C       & R,W,E,C       & R,E           & R,E           \\
    \multicolumn{1}{l|}{\textit{Quality Assurance Evaluators (QA)}}   & $\varnothing$ & $\varnothing$ & $\varnothing$ & O,R,W         & R             & R             & $\varnothing$ & $\varnothing$ & $\varnothing$ & $\varnothing$ & R,E           & R,E           \\
    \multicolumn{1}{l|}{\textit{Maintenance Crew (MC)}}               & $\varnothing$ & $\varnothing$ & $\varnothing$ & $\varnothing$ & O,R,W         & R             & $\varnothing$ & $\varnothing$ & $\varnothing$ & $\varnothing$ & R,E           & R,E           \\
    \multicolumn{1}{l|}{\textit{Flight Data Admin (FDA)}}             & $\varnothing$ & $\varnothing$ & $\varnothing$ & $\varnothing$ & R             & O,R,W         & $\varnothing$ & $\varnothing$ & $\varnothing$ & $\varnothing$ & O,R,W,E,G,C   & O,R,W,E,G,C   \\
    \multicolumn{1}{l|}{\textbf{Remote Navigation Control (RNC)}}     & $\varnothing$ & $\varnothing$ & $\varnothing$ & $\varnothing$ & $\varnothing$ & $\varnothing$ & R,W,E,C       & R             & R             & R             & R             & R             \\
    \multicolumn{1}{l|}{\textbf{Autonomous Navigation Control (ANC)}} & $\varnothing$ & $\varnothing$ & $\varnothing$ & $\varnothing$ & $\varnothing$ & $\varnothing$ & R,W,E,C       & R,W,E,C       & R,W,E,C       & R             & R             & R             \\
    \multicolumn{1}{l|}{\textbf{Navigation Planning (NP)}}            & $\varnothing$ & $\varnothing$ & $\varnothing$ & $\varnothing$ & $\varnothing$ & $\varnothing$ & R             & R             & R             & R             & R             & R             \\
    \multicolumn{1}{l|}{\textbf{Refueling Operation (RO)}}            & $\varnothing$ & $\varnothing$ & $\varnothing$ & $\varnothing$ & $\varnothing$ & $\varnothing$ & $\varnothing$ & $\varnothing$ & R             & R,W,E,C       & R             & R             \\
    \multicolumn{1}{l|}{\textbf{Flight Engine Data (FED)}}            & $\varnothing$ & $\varnothing$ & $\varnothing$ & $\varnothing$ & $\varnothing$ & $\varnothing$ & $\varnothing$ & $\varnothing$ & $\varnothing$ & $\varnothing$ & R,W,E,C       & $\varnothing$ \\
    \multicolumn{1}{l|}{\textbf{Refueling Tank Data (RTD)}}           & $\varnothing$ & $\varnothing$ & $\varnothing$ & $\varnothing$ & $\varnothing$ & $\varnothing$ & $\varnothing$ & $\varnothing$ & $\varnothing$ & $\varnothing$ & $\varnothing$ & R,W,E,C       \\
                                                                      &               &               &               &               &               &               &               &               &               &               &               &              
  \end{tabular}}
  \end{table*}

Our infrastructure security policy breaks down the various actions a system user can perform using the following terms. 

\begin{itemize}
  \item \textit{Subject}: Any entity that contains the proper rights can request the UAS perform operations, access objects, or grant rights to another subject.
  \item \textit{Object}: An entity that is part of the UAS functionality or data that does not have control over another entity. 
  \item \textit{Rights}: A property assigned to a subject that defines its right to access an object or grant permissions to another subject.
\end{itemize}

Table 2 breaks down the subject roles involved in operating the UAS during a refueling mission. Table 1 describes the rights and their associated functionality. Table 3 contains the access control matrix (ACM) showing each subject's rights over the objects.

\subsection{Rights Leakages}

The following basic HRU commands show how given our current ACM we can leak integrity right \textit{r} into a subject that does not contain the right in our intial ACM.
We create a simple command sqeuence that attempts to grant one right from subject \textit{p} in object \textit{o} to subject \textit{q} in \textit{o} if \textit{p} has the given right.

\subsubsection{Confidentiality}

\begin{lstlisting}
command grant_right_to_subject(r, o, p, q)
  if grant in A[p, o] and r in A[p, o]
  then
    enter r into A[q, o];
end
\end{lstlisting}

If we were to call this function with the subject \textit{Pilot Commander (PC)} granting right \textit{Read} to \textit{Maintenance Crew (MC)} for object \textit{Remote Navigation Control (RNC)}, this would cause a confidentiality right leakage. 


\begin{lstlisting}
grant_right_to_subject(Read, RNC, PC, MC)
\end{lstlisting}

\subsubsection{Integrity}

text

% %-------------------------------------------------------------------------
% \section{Survey}
% The various models will be surveyed below.

% %-------------------------------------------------------------------------
% \subsection{HRU}

% \begin{small}
% Access Control Matrix:
% \end{small}

% \begin{figure}[h]
% \begin{tiny}
% \noindent
% \begin{tabular}{|c|c|c|c|c|}
% \hline Objects / Subjects  & Vehicle & Central Authority & Authorized Client & Unauthorized Client \\
% \hline Vehicle &  &  &  &  \\
% \hline Central Authority &  &  &  & \\
% \hline Authorized Client &  & C/D/W & R &  \\
% \hline Unauthorized Client &  & C/D/W &  &  \\
% \hline GPS Location & W &  & R &  \\
% \hline Destination & W &  & R &  \\
% \hline Path & R &  & W &  \\
% \hline

% \end{tabular}
% \end{tiny}
% \caption{ACM Table of the HRU model. Objects vertically and subjects horizontally}

% \end{figure}

% \begin{small}

% Commands that cause rights leakage:

% enter R into ACM[Unauthorized Client, Vehicle];

% enter R into ACM[Unauthorized Client, GPS Location];

% enter R into ACM[Unauthorized Client, Destination];

% enter W into ACM[Unauthorized Client, Path];
% \end{small}

% \subsection{T-G}
% text
% \subsection{BLP}
% text
% \subsection{Biba}
% text
% \subsection{Noninference/Noninterference/ NonDeducibility}
% text
% \section{Specific Models and Rights Leakages}
% \subsection{HRU}
% text
% \subsection{T-G}
% text
% \subsection{BLP}
% text
% \subsection{Biba}
% text
% \subsection{Noninference}
% text
% \subsection{Noninterference}
% text
% \subsection{NonDeducibility}
% text
% \subsection{MSDND}
% text

% \section{Specific Models and Rights Leakages}
% \subsection{HRU}
% text
% \subsection{T-G}
% text
% \subsection{BLP}
% text
% \subsection{Biba}
% text
% \subsection{Noninference}
% text
% \subsection{Noninterference}
% text
% \subsection{NonDeducibility}
% text
% \subsection{MSDND}
% text

% \begin{figure}[h]
%    \caption{Example of caption.}
% \end{figure}

% \section{Conclusions}

% Provide a summative conclusion on the modeling of your infrastructure.

% Can use something like this to put references on a page
% by themselves when using endfloat and the captionsoff option.
\ifCLASSOPTIONcaptionsoff
  \newpage
\fi

\begin{thebibliography}{1}

\bibitem{IEEEhowto:kopka}
A.~Erwin, and J.~Gibson, Navy, Boeing Make Aviation History with MQ-25 Becoming the First Unmanned Aircraft to Refuel Another Aircraft, Accessed on: Sept. 1, 2021. [Online]. Available: https://www.boeing.com/defense/mq25/ 

\end{thebibliography}

% biography section
\begin{IEEEbiographynophoto}{Matthew Whitesides}
  Master's Student at Missouri University of Science and Technology.
\end{IEEEbiographynophoto}

% that's all folks
\end{document}


