% options that should be set.
\documentclass[journal,onecolumn]{IEEEtran}

% correct bad hyphenation here
\hyphenation{op-tical net-works semi-conduc-tor}

\begin{document}

%
% paper title
% Titles are generally capitalized except for words such as a, an, and, as,
% at, but, by, for, in, nor, of, on, or, the, to and up, which are usually
% not capitalized unless they are the first or last word of the title.
% Linebreaks \\ can be used within to get better formatting as desired.
% Do not put math or special symbols in the title.
\title{CS6600 Homework 3}

%
%
% author names and IEEE memberships
% note positions of commas and nonbreaking spaces ( ~ ) LaTeX will not break
% a structure at a ~ so this keeps an author's name from being broken across
% two lines.
% use \thanks{} to gain access to the first footnote area
% a separate \thanks must be used for each paragraph as LaTeX2e's \thanks
% was not built to handle multiple paragraphs
%
\author{Matthew~Whitesides}% <-this % stops a space

% The paper headers
\markboth{Missouri S\&T COMP\_SCI 6600: Formal Methods in CmpSec, Fall~2021}%
{Shell \MakeLowercase{\textit{et al.}}: Bare Demo of IEEEtran.cls for IEEE Journals}

% make the title area
\maketitle

% Note that keywords are not normally used for peerreview papers.
% \begin{IEEEkeywords}
% IEEE, IEEEtran, journal, \LaTeX, paper, template.
% \end{IEEEkeywords}

\IEEEpeerreviewmaketitle

\section{Chapter 3 Problems}

\begin{enumerate}
  \item \(A[s_1,o_1] \; and \; A[s_1,o_2] = A[s_1,o_2] \cup A[s_2,o_2]\) essentailly says that the first two subjects created are able to give the entire scope of rights available to the system as not other subjects with lesser rights would be able to be created after that. This works becase whatever initial rights \textit{s1} had over \textit{o2} would give the rights of any object not explictly created for that object, plus the rights of a subject (\textit{s2}) has over that object.
  If one could test for the absence of rights this statment would still be true however you wouldn't need to use both objects you could just do something like \(A[s_1,o_1] \cup (\forall r \not\in A[s_1,o_1])\) just check for all rights and rights not in that to get the entire scope but that would be more operations.
  \item We can omit the delete and destroy commands because they inhearently cannot "add" rights the can only take them away. Therefore no leak could occur from those commands only the oppiste of a leak which may not be optimal for a user but is secure. This would be different if we could test for the absenece of a right however, given that we could use the test for absense to determine if a right has been deleted from a subject making the minimal set of operations reduced given we currently have to check each right. 
  \item 
  \begin{enumerate}
    \item Modifying the definition to say leaks occur beyond the intial state of the cell, then the delete and destroy commands would affect eh ability to leak a right. If so we cannot get rid of the them because we now have a defintion were you can delete rights potentailly all of a subjects rights, then any addition of any basic right would casuse a leak. 
  \end{enumerate}
  \item First
  \item First
\end{enumerate}

% \appendices
% \section{Proof of the First Zonklar Equation}
% Appendix one text goes here.

% % you can choose not to have a title for an appendix
% % if you want by leaving the argument blank
% \section{}
% Appendix two text goes here.


% use section* for acknowledgment
\section*{Acknowledgment}
The author would like to thank Professor Bruce McMillin with the Department of Computer Science, Missouri University of Science and Technology.

% Can use something like this to put references on a page
% by themselves when using endfloat and the captionsoff option.
\ifCLASSOPTIONcaptionsoff
  \newpage
\fi

% biography section
\begin{IEEEbiographynophoto}{Matthew Whitesides}
  Master's Student at Missouri University of Science and Technology.
\end{IEEEbiographynophoto}

% that's all folks
\end{document}