% options that should be set.
\documentclass[journal,onecolumn]{IEEEtran}

% correct bad hyphenation here
\hyphenation{op-tical net-works semi-conduc-tor}

\begin{document}

%
% paper title
% Titles are generally capitalized except for words such as a, an, and, as,
% at, but, by, for, in, nor, of, on, or, the, to and up, which are usually
% not capitalized unless they are the first or last word of the title.
% Linebreaks \\ can be used within to get better formatting as desired.
% Do not put math or special symbols in the title.
\title{CS6600 Homework 4}

%
%
% author names and IEEE memberships
% note positions of commas and nonbreaking spaces ( ~ ) LaTeX will not break
% a structure at a ~ so this keeps an author's name from being broken across
% two lines.
% use \thanks{} to gain access to the first footnote area
% a separate \thanks must be used for each paragraph as LaTeX2e's \thanks
% was not built to handle multiple paragraphs
%
\author{Matthew~Whitesides}% <-this % stops a space

% The paper headers
\markboth{Missouri S\&T COMP\_SCI 6600: Formal Methods in CmpSec, Fall~2021}%
{Shell \MakeLowercase{\textit{et al.}}: Bare Demo of IEEEtran.cls for IEEE Journals}

% make the title area
\maketitle

% Note that keywords are not normally used for peerreview papers.
% \begin{IEEEkeywords}
% IEEE, IEEEtran, journal, \LaTeX, paper, template.
% \end{IEEEkeywords}

\IEEEpeerreviewmaketitle

\section{Chapter 3 Problems}

\begin{enumerate}
  \item [7)] Lemma 3.1 shows how through the use of a newly created object two subjects with one having take rights over the other, can work together to take a right one of the subjects has over a thrid subject/object. If \textbf{X} was an object however the first step would not work as \textbf{X} could not create the new vertx \textbf{V}. This would not allow \textbf{X} to ever have a connection with an object it has \textit{tg} privliages over so \textbf{Z} would not be able to take the grant privliages from that object and share alpha right with it. 
  \item [9)] Because \(s' = s\:or\:s'\) and \(x_n = s'\) and \(x_i\) are all connected by label \(t,g,bridge\) any of the three options for subject \(x\) will be able to take or pass any right from \(x_n\). Then since there is a sequence of subjects where eventually \(x_n = s'\) and \(s'\) has tg over \(s\) which in turn has \(\alpha\) over \(y\), \(x\) can obtain \(\alpha\) from y. 
  \item [10)]
  \item [11)]
  \item [12)]
\end{enumerate}

% \appendices
% \section{Proof of the First Zonklar Equation}
% Appendix one text goes here.

% % you can choose not to have a title for an appendix
% % if you want by leaving the argument blank
% \section{}
% Appendix two text goes here.


% use section* for acknowledgment
\section*{Acknowledgment}
The author would like to thank Professor Bruce McMillin with the Department of Computer Science, Missouri University of Science and Technology.

% Can use something like this to put references on a page
% by themselves when using endfloat and the captionsoff option.
\ifCLASSOPTIONcaptionsoff
  \newpage
\fi

% biography section
\begin{IEEEbiographynophoto}{Matthew Whitesides}
  Master's Student at Missouri University of Science and Technology.
\end{IEEEbiographynophoto}

% that's all folks
\end{document}