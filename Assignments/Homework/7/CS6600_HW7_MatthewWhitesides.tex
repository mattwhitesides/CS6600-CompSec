% options that should be set.
\documentclass[journal,onecolumn]{IEEEtran}

% correct bad hyphenation here
\hyphenation{op-tical net-works semi-conduc-tor}

\begin{document}

%
% paper title
% Titles are generally capitalized except for words such as a, an, and, as,
% at, but, by, for, in, nor, of, on, or, the, to and up, which are usually
% not capitalized unless they are the first or last word of the title.
% Linebreaks \\ can be used within to get better formatting as desired.
% Do not put math or special symbols in the title.
\title{CS6600 Homework 7}

%
%
% author names and IEEE memberships
% note positions of commas and nonbreaking spaces ( ~ ) LaTeX will not break
% a structure at a ~ so this keeps an author's name from being broken across
% two lines.
% use \thanks{} to gain access to the first footnote area
% a separate \thanks must be used for each paragraph as LaTeX2e's \thanks
% was not built to handle multiple paragraphs
%
\author{Matthew~Whitesides}% <-this % stops a space

% The paper headers
\markboth{Missouri S\&T COMP\_SCI 6600: Formal Methods in CmpSec, Fall~2021}%
{Shell \MakeLowercase{\textit{et al.}}: Bare Demo of IEEEtran.cls for IEEE Journals}

% make the title area
\maketitle

% Note that keywords are not normally used for peerreview papers.
% \begin{IEEEkeywords}
% IEEE, IEEEtran, journal, \LaTeX, paper, template.
% \end{IEEEkeywords}

\IEEEpeerreviewmaketitle

\section{Chapter 8 Problems}

\begin{enumerate}
  \item [1)] Sanitized objects must be in their COI class because sanitized data contains information that is accessible by subjects without creating a COI within the companies data (i.e., if it's public data), so to keep track of that without restricting the data they must be in their class with only sanitized objects. 
  \item [2)] An algorithm to generate an access control matrix \textit{A} from a given history \textit{H} could look something like this. 
  \begin{itemize}
    \item For each subject \textit{s} in \textit{H} append a new row and column to \textit{A}. 
    \item For each object \textit{s} in \textit{H} append a new row and column to \textit{A}. 
    \item This gives us ACM \textit{A[(\#S+\#O),(\#S+\#O)]} set all initial rights to \textbf{allowed}.
    \item For each \textit{h[s,o]} in \textit{H} set all cells in \textit{For each o` in O`: A[s, o`]} rights to \textbf{not allowed} if o is not sanatized. 
    \begin{itemize}
      \item Where \textit{O`} is a set of all values where \textit{$CD(O`) \Rightarrow CD(O)$}.
    \end{itemize}
  \end{itemize}
  \item [3)] While the BLP model cannot show history over time, the CWM can at any iteration support the BLP model. Take a given CWM construction and make the following modifications to support BLP. 
  \begin{itemize}
    \item Sanatized and unsanatized sets represent a security level S for sanatized and U for unsanatized where S dom U.
    \item Each CD and COI object represent categoires a subject has initial access to (i.e. $s1 = (S,\{a, c, b, s\})$).
    \item Each time a subject access an object in a COI class the objects in all other COI classes where \textit{$CD(O`) \Rightarrow CD(O)$} are removed from the subjects set. 
  \end{itemize}
  While this generally gives most subjects a top-level security clearance and each category is its object, it does fit the rules for BLP. Ideally, you may have another set of subcategories to capture the CD and COI sets if you want to convert it back to a BLP model.  
\end{enumerate}

\section{Chapter 9 Problems}

\begin{enumerate}
  \item [3)] Test
  \item [4)] Test
  \item [5)] Test
  \item [6)] Test
\end{enumerate}

% \appendices
% \section{Proof of the First Zonklar Equation}
% Appendix one text goes here.

% % you can choose not to have a title for an appendix
% % if you want by leaving the argument blank
% \section{}
% Appendix two text goes here.


% use section* for acknowledgment
\section*{Acknowledgment}
The author would like to thank Professor Bruce McMillin with the Department of Computer Science, Missouri University of Science and Technology.

% Can use something like this to put references on a page
% by themselves when using endfloat and the captionsoff option.
\ifCLASSOPTIONcaptionsoff
  \newpage
\fi

% biography section
\begin{IEEEbiographynophoto}{Matthew Whitesides}
  Master's Student at Missouri University of Science and Technology.
\end{IEEEbiographynophoto}

% that's all folks
\end{document}