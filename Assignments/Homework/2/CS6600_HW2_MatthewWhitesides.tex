% options that should be set.
\documentclass[journal,onecolumn]{IEEEtran}
\usepackage{hyperref}
\usepackage{listings}
\usepackage{color}

\definecolor{dkgreen}{rgb}{0,0.6,0}
\definecolor{gray}{rgb}{0.5,0.5,0.5}
\definecolor{mauve}{rgb}{0.58,0,0.82}

\lstset{frame=tb,
  language=Python,
  aboveskip=3mm,
  belowskip=3mm,
  showstringspaces=false,
  columns=flexible,
  basicstyle={\small\ttfamily},
  numbers=none,
  numberstyle=\tiny\color{gray},
  keywordstyle=\color{blue},
  commentstyle=\color{dkgreen},
  stringstyle=\color{mauve},
  breaklines=true,
  breakatwhitespace=true,
  tabsize=3,
  morekeywords={if, in, or, then, enter, into, end, command, delete, from, not},
  xleftmargin=\dimexpr 1em+3pt\relax,
  linewidth=\dimexpr \linewidth-3pt\relax
}

\lstnewenvironment{itemlisting}[1][]
 {%
  \mbox{}
  \vspace*{-\baselineskip}
  \lstset{
    xleftmargin=\leftmargin,
    linewidth=\linewidth,
    #1
  }%
 }
 {}

% correct bad hyphenation here
\hyphenation{op-tical net-works semi-conduc-tor}

\begin{document}

%
% paper title
% Titles are generally capitalized except for words such as a, an, and, as,
% at, but, by, for, in, nor, of, on, or, the, to and up, which are usually
% not capitalized unless they are the first or last word of the title.
% Linebreaks \\ can be used within to get better formatting as desired.
% Do not put math or special symbols in the title.
\title{CS6600 Homework 2}

%
%
% author names and IEEE memberships
% note positions of commas and nonbreaking spaces ( ~ ) LaTeX will not break
% a structure at a ~ so this keeps an author's name from being broken across
% two lines.
% use \thanks{} to gain access to the first footnote area
% a separate \thanks must be used for each paragraph as LaTeX2e's \thanks
% was not built to handle multiple paragraphs
%
\author{Matthew~Whitesides \\ \href{mailto:mbwxd4@umsystem.edu}{mbwxd4@umsystem.edu}}% <-this % stops a space

% The paper headers
\markboth{Missouri S\&T COMP\_SCI 6600: Formal Methods in CmpSec, Fall~2021}%
{Shell \MakeLowercase{\textit{et al.}}: Bare Demo of IEEEtran.cls for IEEE Journals}

% make the title area
\maketitle

% Note that keywords are not normally used for peerreview papers.
% \begin{IEEEkeywords}
% IEEE, IEEEtran, journal, \LaTeX, paper, template.
% \end{IEEEkeywords}

\IEEEpeerreviewmaketitle

\section{Chapter 2 Problems}

\begin{enumerate}
  \item [1)] 
    \begin{enumerate} 
      \item 
        \begin{tabular}{lllll}
                         & \textit{alicerc} & \textit{bobrc} & \textit{cyndyrc} &  \\
          \textbf{Alice} & own              & read           & none             &  \\
          \textbf{Bob}   & read             & own            & none             &  \\
          \textbf{Cyndy} & read             & read, write    & own, read, write &  \\
                         &                  &                &                  & 
        \end{tabular}
      \item 
        \begin{tabular}{lllll}
                         & \textit{alicerc} & \textit{bobrc} & \textit{cyndyrc} &  \\
          \textbf{Alice} & own              & read           & read             &  \\
          \textbf{Bob}   & none             & own            & none             &  \\
          \textbf{Cyndy} & read             & read, write    & own, read, write &  \\
                         &                  &                &                  & 
        \end{tabular}
    \end{enumerate} 
  \item [4)] 
    \begin{enumerate} 
\item Remove all q from o.
\begin{lstlisting}
command delete_all_rights(p, q, o)
        delete read from A[q, o];
        delete write from A[q, o];
        delete execute from A[q, o];
        delete append from A[q, o];
        delete list from A[q, o];
        delete modify from A[q, o];
        delete own from A[q, o];
end
\end{lstlisting}
\item Remove all q from o if p can modify o.
\begin{lstlisting}
  command delete_all_rights(p, q, o)
          if modify in A[p, o]
          then 
            delete read from A[q, o];
            delete write from A[q, o];
            delete execute from A[q, o];
            delete append from A[q, o];
            delete list from A[q, o];
            delete modify from A[q, o];
            delete own from A[q, o];
  end
\end{lstlisting}
\item Remove all q from o if p can modify o and q does not have own rights.
\begin{lstlisting}
  command delete_all_rights(p, q, o)
          if modify in A[p, o] and own not in A[q, o]
          then 
            delete read from A[q, o];
            delete write from A[q, o];
            delete execute from A[q, o];
            delete append from A[q, o];
            delete list from A[q, o];
            delete modify from A[q, o];
            delete own from A[q, o];
  end
\end{lstlisting}
    \end{enumerate}
  \item [5)]
    \begin{enumerate} 
\item Copy all rights from p to q.
\begin{lstlisting}
command copy_all_rights(p, q, o)
  if read in A[p, o]
  then
    enter read into A[q, o];
  if write in A[p, o]
  then
      enter write into A[q, o];
  if execute in A[p, o]
  then
      enter execute into A[q, o];
  if append in A[p, o]
  then
      enter append into A[q, o];
  if list in A[p, o]
  then
      enter list into A[q, o];
  if modify in A[p, o]
  then
      enter modify into A[q, o];
  if own in A[p, o]
  then
      enter list into A[q, o];
end
\end{lstlisting}
\item Copy all rights from p to q if the right has the copy flag.
\begin{lstlisting}
command copy_all_rights(p, q, o)
  if read in A[p, o] and copy in A[p, o]
  then
    enter read into A[q, o];
  if write in A[p, o] and copy in A[p, o]
  then
      enter write into A[q, o];
  if execute in A[p, o] and copy in A[p, o]
  then
      enter execute into A[q, o];
  if append in A[p, o] and copy in A[p, o]
  then
      enter append into A[q, o];
  if list in A[p, o] and copy in A[p, o]
  then
      enter list into A[q, o];
  if modify in A[p, o] and copy in A[p, o]
  then
      enter modify into A[q, o];
  if own in A[p, o] and copy in A[p, o]
  then
      enter list into A[q, o];
end
\end{lstlisting}
\item If you copied the copy flag along with the right you would essentially be giving the copied user the ability to copy those rights over to any other user of their choosing. 
    \end{enumerate}
  \item [6)] 
  \begin{enumerate} 
    \item Copy r rights from Alice (p) to book to Bob (q).
    \begin{lstlisting}
    command copy_right_to_book(r, p, q)
      if c in A[p, book] and r in A[p, book]
      then
        enter r into A[q, book];
    end
    \end{lstlisting}
    \item Copy r rights from Alice (p) to book to Bob (q).
    \begin{lstlisting}
    command copy_right_to_book(r, p, q)
      if rc in A[p, book]
      then
        enter r into A[q, book];
    end
    \end{lstlisting}    
    \item If the copy flag is not copied over to Bob, Bob would not have the ability to copy right r over to anyone else. Which I would think would be correct that you would not want to give bob rc by default that might be a separate command. 
        \end{enumerate}
  \item [10)] 
  \begin{enumerate}
    \item If you did not apply the principle of attenuation at all in a system, any subject would be able to grant rights or increase their own rights by granting them to themselves. Essentially all the rights in the system would be available to any subject, this may only be limited by what rights the subject is aware they could have, so if they could cooperate with other subjects they would potentially be able to have the rights of all unique rights over the set of all subjects. Also if there is a copy ability any subject with read permission could copy a file and grant anyone else full permissions as they would be the owner of the copy. 
    \item If attenuation was applied to access rights but not own and grant\_rights, this would not help the situation as if the subjects with the grant or own permissions were in on it they have the ability to grant any permission to any other subject regardless if they have the right themselves. If the subjects with the grant permission did not have the ability to grant permissions that they themselves do not have like read then that would limit the maximal set to the rights of any subject union with the subjects with grant permission.
    \item The following ACM shows two subjects with subject1 being an ancestor of subject2, subject1 has right r which is inherited by subject2. This would lead to the maximal set being the union of any given subject and all its ancestors. \\\\
    \begin{tabular}{lllll}
    & \textit{subject1} & \textit{subject2} & \textit{object} &  \\
    \textbf{subject1} & r                 & none              & r               &  \\
    \textbf{subject2} & r                 & r                 & r               & 
    \end{tabular}

  \end{enumerate}
\end{enumerate}

% \appendices
% \section{Proof of the First Zonklar Equation}
% Appendix one text goes here.

% % you can choose not to have a title for an appendix
% % if you want by leaving the argument blank
% \section{}
% Appendix two text goes here.


% use section* for acknowledgment
\section*{Acknowledgment}
The author would like to thank Professor Bruce McMillin with the Department of Computer Science, Missouri University of Science and Technology.

% Can use something like this to put references on a page
% by themselves when using endfloat and the captionsoff option.
\ifCLASSOPTIONcaptionsoff
  \newpage
\fi

% biography section
\begin{IEEEbiographynophoto}{Matthew Whitesides}
  Master's Student at Missouri University of Science and Technology.
\end{IEEEbiographynophoto}

% that's all folks
\end{document}